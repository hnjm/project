\section{Datenquellen und Aufbereitung}

Als grundlegende Datenquelle wurden die
\textit{color FERET Database}\footnote{NIST, color FERET Database.\newline(https://www.nist.gov/itl/iad/image-group/color-feret-database)\label{footnote:color-feret}},
die von dem National Institute of Standards and Technology veröffentlicht wurde, und die
Datenbank \textit{FaceScrub} von \textit{vintage}\footnote{vision \& interaction groupe, FaceScrub.\newline(http://vintage.winklerbros.net/facescrub.html\label{footnote:face-scrub}} verwendet.

Die \textit{color FERET Database}\textsuperscript{\ref{footnote:color-feret}} umfasst 11.338 Gesichtsbilder von 1.208 Menschen, die in einer kontrollierten Umgebung augenommen worden sind, und
hält neben Metadaten über Pose, Geschlecht, Ethnie und Alter auch weiterführende Daten bereit wie Augenposition und
Kamerawinkel. Die Daten liegen im Portable Pixmap Format RGB- und im Graustufenformat homogen in einer vor Maximalauflösung von \textit{512 \(\times\) 768} Pixeln vor.

Die \textit{FaceScrub}\textsuperscript{\ref{footnote:face-scrub}} Datenbank umfasst über 100.000 Gesichtsbilder von 530 prominenten Menschen, die in einer unkontrollierten Umgebung augenommen worden sind und stellt keine
weiteren Metadaten bereit. Die Daten liegen im JPEG-Format in einer vor Maximalauflösung von \textit{512 \(\times\) 768} Pixeln vor.


Um trotz der vergleichsweise geringen Datenmenge. Vergleichbare Projekte\footnote{Richard McPherson, Rezar Shokri, Vitali Shmatikov, Defeating Image Obfuscation with Deep Learning.\newline(https://arxiv.org/pdf/1609.00408.pdf)}
verwenden hingegen 60.000 bis 300.000 Bilder, interpretierbare Ergebnisse erzielen zu können, beschränkt sich diese Arbeit auf
die Verwendung möglichst homogener Bilder unterschiedlicher Personen. Von besonderem Interesse ist hierbei die Pose des
Abgebildeten. Die Datenbank unterscheidet Frontal- und Profilbilder sowie Bilder, in denen der Kopf um einen bestimmten
Winkel gedreht ist. Als grundlegenden Datensatz wurde sich für die Frontalbilder entschieden, da diese mit 2.722 Bilder
von 994 Personen den großten Teildatensatz ausmachen.

Die benötigten Testdatensätze wurde mithilfe von ImageMagick\footnote{ImageMagick Studio LLC, ImageMagick.\newline(https://www.imagemagick.org/)} in Version 6.8. aufbereitet. Um die Komplexität der
Problemstellung weiter zu reduzieren, wurden die Bilder grauskaliert und auf 12,5\% der Ursprungsgröße skaliert, sodass
die Trainigsdaten noch eine Auflösung von 64 \(\times\) 96 Pixeln haben. Es wurden vier unterschiedliche Testdatensätze mit
folgenden CLI-Befehlen generiert \footnote{Das folgende BASH-Skript scripts/create\_images.sh erzeugt die
Testdaten. Notwendig hierfür sind die Pakete \textit{imagemagick} und \textit{imagemagick-doc}.}:


\begin{minipage}{\linewidth}

\begin{lstlisting}[language=bash,caption={convert - Synopsis}]
convert [input-options] input-file [output-options] output-file
\end{lstlisting}

\begin{lstlisting}[language=bash,caption={Testdatenerstellung - Graustufen}]
#!/bin/bash

# every call scales the input image down to 12.5% of its
# original size and grayscales it.

# convert test data:  gaussian-blur (sigma = 3)
convert <input_file.ppm> \
    -set colorspace Gray \  # grayscale durch
    -separate \             # separates (RGB)
    -average \              # Mittel
    -scale 12.5\% \         # Skalierung auf 96px x 64px
    -gaussian-blur 0x3 \    # blur
    <output_file.pgm>; mv <output_file.pgm> <output_file.ppm>

# convert test data:  gaussian-blur (sigma = 6)
convert <input_file.ppm> \
    -set colorspace Gray \
    -separate \
    -average \
    -scale 12.5\% \
    -gaussian-blur 0x6 \
    <output_file.pgm>; mv <output_file.pgm> <output_file.ppm>

# convert test data:  pixelization (edge length = 5px)
convert <input_file.ppm> \
    -set colorspace Gray \
    -separate \
    -average \
    -scale 12.5\% \
    -scale $(( bc <<< "scale=100;100/5" ))\% \
    -scale 500\% \
    <output_file.pgm>; mv <output_file.pgm> <output_file.ppm>

# convert test data:  pixelization (edge length = 10px)
convert <input_file.ppm> \
    -set colorspace Gray \
    -separate \
    -average \
    -scale 12.5\% \
    -scale $(( bc <<< "scale=100;100/10" ))\% \
    -scale 1000\% \
    <output_file.pgm>; mv <output_file.pgm> <output_file.ppm>
\end{lstlisting}
\end{minipage}